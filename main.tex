\documentclass[11pt,A4]{article}
\usepackage[utf8]{inputenc}
\usepackage[english]{babel}

\usepackage{xcolor}
\definecolor{ArchCode}{HTML}{EBF1F5}
\definecolor{ArchLink}{HTML}{0077BB}
\definecolor{ArchGreen}{HTML}{DDFFDD}
\usepackage{tcolorbox}

\usepackage[colorlinks = true,
            linkcolor =ArchLink,
            urlcolor  = ArchLink,
            citecolor =ArchLink,
            anchorcolor = ArchLink 
            ]{hyperref}

\usepackage{minted}
%% se https://www.overleaf.com/learn/latex/Code_Highlighting_with_minted#Reference_guide for other styles
\usemintedstyle{pastie}

\usepackage{geometry}
\geometry{a4paper,portrait,margin=2cm}

\title{Arch install for DELL XPS 9570}
\author{Simone Ruffini }
\date{March 2020}

\newcommand{\linecode}[1]{ \mint[fontsize=\small,bgcolor=ArchCode,frame=single]{bash}|#1|}

\begin{document}

\maketitle
Read everitying while keeping an eye to \href{https://wiki.archlinux.org/index.php/Installation_guide}{\textbf{Arch installation Guide}} and \href{https://www.gloriouseggroll.tv/arch-linux-efi-install-guide/}{GloriusEggroll}
\section{Pre-Installation}
\subsection{Create Bootable Media}
Create a bootable device with RUFUS with dd method () or by using dd.
\begin{itemize}
    \item Windows: RUFUS with dd
    \item *nix:
\end{itemize}

\linecode {sudo dd bs=4M if=<path/to/input>.iso of=/dev/sd<?> conv=fdatasync  status=progress}
Boot in arch linux. Remember to disable Secure Boot and RAID intel otherwise nvme devices wont be shown.
\subsection{Verify the boot mode}
If \href{https://wiki.archlinux.org/index.php/UEFI}{UEFI} mode is enabled on an \href{https://wiki.archlinux.org/index.php/UEFI}{UEFI} motherboard, Archiso will \href{https://wiki.archlinux.org/index.php/Boot}{boot} Arch Linux accordingly via \href{https://wiki.archlinux.org/index.php/Systemd-boot}{systemd-boot}. To verify this, list the efivars directory:
\linecode{ls /sys/firmware/efi/efivars}

\vspace{-5mm}
\subsection{Connect to the internet}
To set up a network connection, go through the following steps:
\newline
Ensure your \href{https://wiki.archlinux.org/index.php/Network_interface}{network} interface is listed and enabled, for example with \href{https://jlk.fjfi.cvut.cz/arch/manpages/man/ip-link.8}{ip-link(8)}:
\linecode{ip link}
\href{https://wiki.archlinux.org/index.php/Wireless_network_configurationc}{Connect to the wireless LAN}.
Try with wifi-menu of \href{https://wiki.archlinux.org/index.php/Netctl}{netctl} package:
\linecode{wifi-menu}
\vspace{-5mm}
\begin{tcolorbox}[colback=ArchGreen,sharp corners,boxrule=0.2mm]
If somethig goes wrong you can modify the network profile with \\ \texttt{vim /etc/netctl/<profile\_name>}
\end{tcolorbox}
The connection may be verified with \href{https://en.wikipedia.org/wiki/ping}{ping}:
\linecode{ping archlinux.org}

\subsection{Update the system clock}
Use \href{https://jlk.fjfi.cvut.cz/arch/manpages/man/timedatectl.1}{timedatectl(1)} to ensure the system clock is accurate:

\linecode{timedatectl set-ntp true}

\vspace{-5mm}
\subsection{Partition the disks}
When recognized by the live system, disks are assigned to a \href{https://wiki.archlinux.org/index.php/Device_file#Block_devices}{block device} such as \texttt{/dev/sda }or \texttt{/dev/nvme0n1}. To identify these devices, use \href{https://wiki.archlinux.org/index.php/Lsblk}{lsblk} or \href{https://wiki.archlinux.org/index.php/Fdisk}{fdisk}.

\linecode{lsblk}
Results ending in \texttt{rom}, \texttt{loop} or \texttt{airoot} may be ignored.
The following \href{https://wiki.archlinux.org/index.php/Partition}{partitions} are \textbf{required} for a chosen device:
\begin{itemize}
    \item One partition for the root directory /.
    \item If \href{https://wiki.archlinux.org/index.php/UEFI}{UEFI} is enabled, an \href{https://wiki.archlinux.org/index.php/EFI_system_partition}{EFI system partition}.
\end{itemize}
Using \href{https://wiki.archlinux.org/index.php/LVM}{LVM} and with
a graphical interface to partiton disk use \href{https://jlk.fjfi.cvut.cz/arch/manpages/man/cfdisk.8}{cfdisk}.
\linecode{cfdisk /dev/<device>}
Inside cfdisk:
\begin{enumerate}
    \item Make a GPT partition table if not already exisiting
    \item New 512M partition for EFI, change partition type with \texttt{EFI System}
    \item Remaining of disk size (no option) with default type \texttt{Linux filesystem} 
    \item write and quit 
\end{enumerate}
\subsubsection{LVM partitions}
Visit \href{https://wiki.archlinux.org/index.php/LVM_(Italiano)}{LVM ita}. \textbf{In this section the storage device will be sda}!
\begin{enumerate}
    \item Create physical volume and check with \texttt{pvdisplay}
    \vspace{-2mm}\linecode{pvcreate /dev/sda2} \vspace{-10mm}
    \item Create Volume Group and check with \texttt{vgdisplay}
    \vspace{-2mm}\linecode{vgcreate VolGrp0 /dev/sda2}\vspace{-10mm}
    \item Create Logical Volumes and check with \texttt{lvldisplay} 
    \vspace{-2mm}
    \begin{minted}[fontsize=\footnotesize,bgcolor=ArchCode,frame=single]{bash}
lvcreate -L 20G VolGrp0 -n root
lvcreate -C y -L 8G VolGrp0 -n swap
lvcreate -l 100%FREE VolGrp0 -n home
    \end{minted}
\end{enumerate}

\vspace{-5mm}
\subsection{Format the partitions}
Once the partitions have been created, each must be formatted with an appropriate \href{https://wiki.archlinux.org/index.php/File_system}{file system}.

\begin{minted}[fontsize=\footnotesize,bgcolor=ArchCode,frame=single]{bash}
mkfs.fat -F 32 /dev/<boot_partition> 
mkfs.f2fs -l root /dev/VolGrp00/root
mkfs.f2fs -l home /dev/VolGrp0/home
\end{minted}
Initialize the swap
\begin{minted}[fontsize=\footnotesize,bgcolor=ArchCode,frame=single]{bash}
mkswap /dev/VolGrp0/swap
swapon /dev/VolGrp0/swap
\end{minted}
\begin{tcolorbox}[boxrule=0.2mm]
The partitioning scheme and mount point of the EFI Sytem Partition (ESB) are tided to the type of booloader used. In this guide \href{https://wiki.archlinux.org/index.php/Systemd-boot}{systemd-boot} will be used (look at installation section).
The mount point of the ESB partition in Systemd-boot needs to contain kernel and initramfs files. So boot is effectively the ESP. 
\end{tcolorbox}

\subsection{Mount the file systems}
Non existen directories must be created first
\begin{minted}[fontsize=\footnotesize,bgcolor=ArchCode,frame=single]{bash}
mount /dev/VolGrp0/root /mnt
mkdir /mnt/home
mkdir /mnt/boot
mount /dev/VolGrp0/home /mnt/home
mount /dev/<boot_partition> /mnt/boot
\end{minted}
\section{Installation}
\subsection{Select mirrors}
Packages to be installed must be downloaded from \href{https://wiki.archlinux.org/index.php/Mirrors}{mirror servers}, which are defined in \\ \texttt{/etc/pacman.d/mirrorlist}. The higher a mirror is placed in the list, the more priority it is given when downloading a package. This file will later be copied to the new system by pacstrap, so it is worth getting right.\\
Make a backup of the mirror list:
\linecode{cp /etc/pacman.d/mirrorlist /etc/pacman.d/mirrorlist.backup}
Install \href{https://git.archlinux.org/pacman-contrib.git/about/}{pacman-contrib} package containing the \texttt{rankmirrors} script.
\begin{minted}[fontsize=\footnotesize,bgcolor=ArchCode,frame=single]{bash}
pacman -Sy
pacman -S pacman-contrib
\end{minted}
\begin{tcolorbox}[colback=ArchGreen,sharp corners,boxrule=0.2mm]
If pacman is not working then probably all servers in the mirror list are commented. Uncomment one by modyfing \texttt{mirrorlist} and after the install recomment it.
\end{tcolorbox}
\noindent To be shure that all servers are availble for ranking run tht command to uncomment all the lines in \texttt{mirrorlist}:
\mint[fontsize=\footnotesize,bgcolor=ArchCode,frame=single]{html}|sed -i 's/^#Server/Server/' /etc/pacman.d/mirrorlist.backup|
Now we will run \texttt{mirrors}, it can take a wile and no output is made in the process so just wait and monitor on another tty with \texttt{top}. The script will rank the first 6 best mirrors delting the other ones.
\vspace{-1mm}
\linecode{rankmirrors -n 6 /etc/pacman.d/mirrorlist.backup > /etc/pacman.d/mirrorlist}
If you want add the other servers for precaution.

\subsection{Install essential packages}
Use the \href{https://projects.archlinux.org/arch-install-scripts.git/tree/pacstrap.in}{pacstrap} script to install the \href{https://www.archlinux.org/packages/?name=base}{base} package, Linux \href{https://wiki.archlinux.org/index.php/Kernel}{kernel} and firmware for common hardware.
The \href{https://www.archlinux.org/packages/?name=base}{base} package does not include all tools from the live installation, so installing other packages may be necessary for a fully functional base system. In particular, consider installing:
\begin{itemize}
    \item userspace utilities for the management of \href{https://wiki.archlinux.org/index.php/File_systems}{file systems} that will be used on the system,
    \item utilities for accessing \href{https://wiki.archlinux.org/index.php/RAID}{RAID} or \href{https://wiki.archlinux.org/index.php/LVM}{LVM} partitions,
    \item specific firmware for other devices not included in \href{https://www.archlinux.org/packages/?name=linux-firmware}{linux-firmware},
    \item software necessary for \href{https://wiki.archlinux.org/index.php/Networking}{networking},
    \item a \href{https://wiki.archlinux.org/index.php/Text_editor}{text editor},
\item packages for accessing documentation in \href{https://wiki.archlinux.org/index.php/Man}{man} and \href{https://wiki.archlinux.org/index.php/Info}{info} pages: \href{https://www.archlinux.org/packages/?name=man-db}{man-db}, \href{https://www.archlinux.org/packages/?name=man-pages}{man-pages} and \href{https://www.archlinux.org/packages/?name=texinfo}{texinfo}.
\end{itemize}
    
\noindent To \href{https://wiki.archlinux.org/index.php/Install}{install} other packages or package groups, append the names to the pacstrap command above (space separated) or use \href{https://wiki.archlinux.org/index.php/Pacman}{pacman} while \href{https://wiki.archlinux.org/index.php/Installation_guide#Chroot}{chrooted into the new system}. For comparison, packages available in the live system can be found in \href{https://projects.archlinux.org/archiso.git/tree/configs/releng/packages.x86_64}{packages.x86\_64}.
\linecode{pacstrap /mnt base base-devel linux linux-firmware vim man-db man-pages lvm2}

\vspace{-5mm}
\section{Configure the system}

\subsection{Generate Fstab}
\vspace{-3mm}
\linecode{genfstab -U -p /mnt >> /mnt/etc/fstab}
Check if there is a entry for every partition and the one for swap too with \texttt{vim}.
\subsection{Chroot}
Now we are going to \href{https://wiki.archlinux.org/index.php/Change_root}{chroot} into our newly installed system and begin to configure its booting, time, and language:
\linecode{arch-chroot /mnt}

\vspace{-7mm}
\subsection{Time zone}
Set the \href{https://wiki.archlinux.org/index.php/Time_zone}{time zone}:
\linecode{ ln -sf /usr/share/zoneinfo/Europe/Rome /etc/localtime}
Run \href{https://jlk.fjfi.cvut.cz/arch/manpages/man/hwclock.8}{hwclock(8)} to generate \texttt{/etc/adjtime}:
\linecode{hwclock --systohc --utc}

\vspace{-7mm}
\subsection{Localization}
Uncomment \texttt{en\_US.UTF-8 UTF-8} and other needed \href{https://wiki.archlinux.org/index.php/Locale}{locale} in \texttt{/etc/locale.gen}, and generate them with:
\linecode{locale-gen}
Create the \href{https://jlk.fjfi.cvut.cz/arch/manpages/man/locale.conf.5}{locale.conf(5)} file, set the \texttt{LANG} \href{https://wiki.archlinux.org/index.php/Variable}{variable} as your language:
\begin{minted}[fontsize=\footnotesize,bgcolor=ArchCode,frame=single]{bash}
echo LANG=en_US.UTF-8 > /etc/locale.conf
export LANG=en_US.UTF-8
\end{minted}

\vspace{-2mm}
\subsection{Network configuration}
\subsubsection{Hostname}
Create the \href{https://wiki.archlinux.org/index.php/Hostname}{hostname} file:
\linecode{echo DellXPS-Simone > /etc/hostname}
Add matching entries to \href{https://jlk.fjfi.cvut.cz/arch/manpages/man/hosts.5}{hosts(5)}:
\begin{minted}[fontsize=\footnotesize,bgcolor=ArchCode,frame=single]{bash}
vim /etc/hosts
----------------------------------------------------------------------------------------------------
127.0.0.1	localhost
::1		    localhost
127.0.1.1	DellXPS-Simone.localdomain	DellXPS-Simone
\end{minted}
\subsubsection{Network Manager}
Install network tools and mangaer:
\linecode{pacman -S NetworkManager}
Find any network devices starting with \texttt{enp/wlp} with:
\linecode{ip link}
Disable \texttt{DHCPCD} and \texttt{netctl} on any network device since\href{https://wiki.archlinux.org/index.php/NetworkManager}{networkmanager} will replace both.\\
\vspace{-3mm}
\begin{minted}[fontsize=\footnotesize,bgcolor=ArchCode,frame=single]{bash}
systemctl disable dhcpcd@<eth_device>.service
systemctl disable netctl-auto@<wirl_device>.service
\end{minted}
Enable \href{https://wiki.archlinux.org/index.php/NetworkManager}{networkmanager}
\linecode{systemctl enable NetworkManager.service}

\vspace{-7mm}
\subsection{Enable trim support}
For safe, weekly \href{https://wiki.archlinux.org/index.php/Solid_state_drive#TRIM}{TRIM} service on \href{https://wiki.archlinux.org/index.php/Solid_state_drive}{SSDs} and all other devices that enable \href{https://wiki.archlinux.org/index.php/Solid_state_drive#TRIM}{TRIM} support:
\linecode{systemctl enable fstrim.timer}

\vspace{-7mm}
\subsection{Enabling multilib and Arch AUR}
If you are running a 64bit system then you need to enable the \href{https://wiki.archlinux.org/index.php/Official_repositories#multilib}{multilib repository}.\\
Uncomment in \texttt{/etc/pacman.conf}: 
\begin{minted}[fontsize=\footnotesize,bgcolor=ArchCode,frame=single]{bash}
[multilib]
Include = /etc/pacman.d/mirrorlist
\end{minted}
Update the sistem with pacman.

\subsection{Boot loader}
Install \href{https://wiki.archlinux.org/index.php/Systemd-boot}{systemd-boot} \href{https://wiki.archlinux.org/index.php/Boot_loader}{bootloader}
Recheck if the EFI variables are mounted
\linecode{mount -t efivarfs efivarfs /sys/firmware/efi/efivars}
With the ESP mounted to \texttt{/boot}, use \href{https://jlk.fjfi.cvut.cz/arch/manpages/man/bootctl.1}{bootctl(1)} to install systemd-boot into the EFI system partition by running:
\linecode{bootctl --path=/boot install}
This will copy the systemd-boot boot loader to the EFI partition, it will then set systemd-boot as the default EFI application (default boot entry) loaded by the EFI Boot Manager.\\
Create configuration file to add an entry for Arch Linux:
\begin{minted}[fontsize=\footnotesize,bgcolor=ArchCode,frame=single]{bash}
vim /boot/loader/entries/arch.conf
----------------------------------------------------------------------------------------------------
title Arch Linux
linux /vmlinuz-linux
initrd /initramfs-linux.img
\end{minted}
Next add the \texttt{root} partition to the loader entry, since ours is \texttt{/dev/VolGrp0/root}:
\begin{minted}[fontsize=\footnotesize,bgcolor=ArchCode,frame=single]{bash}
echo "options root=PARTUUID=$(blkid -s PARTUUID -o value /dev/VolGrp0/root) rw" >>
/boot/loader/entries/arch.conf
\end{minted}

\subsection{Add Intel microcode}
Install \href{https://wiki.archlinux.org/index.php/Microcode}{microcode}:
\linecode{pacman -S intel-ucode}
Update the arch systemd-boot loader file to load microcode:
\begin{minted}[fontsize=\footnotesize,bgcolor=ArchCode,frame=single]{bash}
vim /boot/loader/entries/arch.conf
----------------------------------------------------------------------------------------------------
...
initrd /intel-ucode.img
initrd /initramfs-linux.img
...
\end{minted}

\subsection{Root password} 
Set the root \href{https://wiki.archlinux.org/index.php/Password}{password}:
\linecode{passwd}

\vspace{-7mm}
\subsection{User setup} 
Add a default \href{https://wiki.archlinux.org/index.php/Users_and_groups}{user} with:
\linecode{useradd -m -g users -G wheel,storage,power -s /bin/bash simone}
set a \href{https://wiki.archlinux.org/index.php/Password}{password} for that user:
\linecode{passwd simone} 

\vspace{-7mm}
\subsection{Setting up sudoers}
Edit the sudoers file to give this user \href{https://wiki.archlinux.org/index.php/Sudo#Configuration}{sudo} privileges. 
The configuration file for sudo is in \texttt{/etc/sudoers}. It should always be edited with the \href{https://jlk.fjfi.cvut.cz/arch/manpages/man/visudo.8}{visudo(8)} command. \texttt{visudo} locks the \texttt{sudoers} file, saves edits to a temporary file, and checks that file's grammar before copying it to \texttt{/etc/sudoers}.\\
Uncomment:
\linecode{wheel ALL=(ALL) ALL}
\noindent Make sudoers require typing the root password instead of their own password by adding:
\linecode{Defaults rootpw}

\subsection{LVM checks}
Check that \texttt{lvm1}, \texttt{udev} hooks and \texttt{dm\_mod} module are enabled in \href{https://wiki.archlinux.org/index.php/Mkinitcpio}{mkinitcpio}:

\begin{minted}[fontsize=\footnotesize,bgcolor=ArchCode,frame=single]{bash}
vim /minted/mkinitcpio.conf
----------------------------------------------------------------------------------------------------
HOOKS="base udev ... lvm2 filesystems"
...
MODULES="dm_mod..."
\end{minted}
if changes to the file are made, \texttt{initramfs} has to be \href{https://wiki.archlinux.org/index.php/Mkinitcpio#Installation}{rebuilt}.\\
Kernel option \texttt{dolvm} could be passed.
\subsection{Install additional Packages}
\linecode{pacman -S bash-completion }
add systemd pacman hook to update systemd
\end{document}
